\section*{Introduction}

This course introduces digital logic and its application in computer organisation and design.

The major emphasis is on practical design work. In the laboratory state-of-the-art computer-aided design tools are used to support the design of digital hardware systems. Students' designs are simulated and then implemented on in-house programmable gate array boards.

The lectures initially support the laboratories but progress to a wider overview of the design and interaction of computer hardware systems. Ultimately a complete - if simple - computer is described.

\section*{Aims}

The main aim of this course is to give students a basic understanding of the hardware which underpins computing systems.

Further aims include:

\begin{itemize}
	\item Introduction to basic logic and logic gates
	Partitioning of simple systems into combinatorial and sequential blocks.
	\item To introduce basic CAD tools to aid in the design of a basic computer system
	\item To provide an overview of hardware description languages with particular emphasis on Verilog
	\item Introducing logic level implementation of a simple processor
	\item Discussion of how computer systems interact with memory and I/O devices
\end{itemize}